\documentclass[a4paper,twoside]{article}

\usepackage{epsfig}
\usepackage{subfigure}
\usepackage{calc}
\usepackage{amssymb}
\usepackage{amstext}
\usepackage{amsmath}
\usepackage{amsthm}
\usepackage{multicol}
\usepackage{pslatex}
\usepackage{apalike}
\usepackage{SCITEPRESS}     % Please add other packages that you may need BEFORE the SCITEPRESS.sty package.

\subfigtopskip=0pt
\subfigcapskip=0pt
\subfigbottomskip=0pt

\begin{document}


\title{GPU Virtualization in the Cloud (a mejorar el t\'itulo!!!)}
\author{\authorname{Sergio Iserte, Francisco J. Clemente-Castell\'o, Adri\'an Castell\'o,\\Rafael Mayo and Enrique S. Quintana-Ort\'i}
\affiliation{Department of Computer Science and Engineering\\Universitat Jaume I - Castell\'o de la Plana, Spain}
\email{\{siserte, fclement, adcastel, mayo, quintana\}@uji.es}}

\keywords{Cloud Computing, GPU Virtualization, Resource Management}

\abstract{The popularity of the GPGPU computation has led several cloud vendors to provide instances of virtual machines with GPUs. 
So that, the guest hosts must be equipped with GPUs which could be barely utilized if no GPU-enabled VM  is running in the host.
The solution presented here is based on GPU virtualization and shareability in order to reach an equilibrium between supply and demand of accelerators from the users. 
Hence, we propose to access any GPU from any VM getting rid of the necessity of having physical GPUs in the guest host. 
Finally, our project goes further and it takes into account that by making logical partitions of the GPU memory we could address each partition as if it was an independent GPU device.}

\onecolumn \maketitle \normalsize \vfill

\section{\uppercase{Introduction}}
\label{sec:introduction}

\noindent 
Nowadays, to provide a virtual machine with GPUs means
that the guest host must be equipped with at least 1 GPU, and
once it has been assigned to an instance, no other VM can use
it. Depending on the users’ necessities, two approaches to introduce GPGPU computation in the cluster can be determined:
the conservative, to have a lot of hosts with GPUs to be
ensured that you are likely to satisfy all the requests; or a
more daring approach, where you will count with some GPU-
enabled nodes hoping not to arrive a large burst of requests
so the cluster would run out of GPU resources. Our solutions
is based on GPU virtualization and sharing resources in order
to reach a fair balance between supply and demand.
Several attempts of achieving what we are pursing have been
made, but none of them are so ambitious as our project. On
the one hand the work done in [1] allows the VM managed
by the hypervisor Xen to access the GPUs in the physical
node. With the implied limitations of restricting the nodes not
to use more GPUs than the hosted in the machine and not
to share idle GPUs to other machines. The solution presented
by the project gVirtus [2] does virtualize GPUs and make
them accessible for any VM in the cluster. However, this kind
of virtualization strongly depends on the hypervisor used, so
does its performance. Another similar solution is presented in
[3] by the name of gCloud. This one continues without being
integrated in a Cloud Computing Manager, though its main
drawback is that the code of the applications must be modified
in order to be run in their virtual-GPU environment. On the
other hand, the work exposed in [4] is more mature, however,
it is only focussed on compute intensive HPC applications.
The main idea of our proposal goes further and apart from
bringing solutions for all kind of HPC applications, it is aimed
to boost the cluster flexibility in the use of GPUs.

\section{\uppercase{Background}}
\label{sec:background}
\subsection{The rCUDA Framework}
\label{sec:rcuda}

{rCUDA}~\cite{tonithesis,toniparco} is a middleware that enables transparent access
to any NVIDIA GPU device present in a cluster from all compute
nodes. The GPUs can be accessed and shared between nodes, and a single node can use all these graphic accelerators
as if they were local.
These features are focused in attaining higher accelerator utilization rates in the overall system while simultaneously reducing
resource, space, and energy~\cite{energy14} requirements.
rCUDA is structured following a client-server distributed
architecture: the client middleware is allocated in the same cluster node where the application demanding GPGPU
acceleration services is executed, providing a transparent replacement for the
native CUDA libraries. Furthermore, the server middleware is executed in the
cluster nodes from which the actual GPUs provide the requested GPGPU service.
To support a concurrent scenario where GPUs are shared between
processes\slash nodes, {rCUDA} manages separate device contexts for
each client application.

The {rCUDA} 5.0 client exposes the same interface as the regular NVIDIA
CUDA 6.5 release~\cite{cuda65}, including the runtime and driver
APIs as well as other commonly used libraries such as cuBLAS, cuFFT, cuSparse or cuRand.
Therefore, applications are not aware of the fact that they are being executed
on top of a virtualization layer.
With the aim to be updated with new GPU programming models, {rCUDA} also supports
directive-based models such as OmpSS~\cite{repara15} and OpenACC~\cite{cluster15}.

The integration of remote GPGPU virtualization with global
resource schedulers such as SLURM~\cite{sbacpad14} completes this powerful
technology, making accelerator-enabled clusters more flexible and
energy efficient.

\subsection{Manuscript Setup}


\subsection{Page Setup}


\section*{\uppercase{Acknowledgements}}

\bibliographystyle{apalike}
{\small
\bibliography{paper}}


\end{document}

